\documentclass[12pt,a4paper]{article}

% 中文支援套件
\usepackage[UTF8,scheme=chinese]{ctex}
\usepackage{xeCJK}

% 繁體中文設定
\ctexset{
    contentsname={目錄},
    listfigurename={圖目錄},
    listtablename={表目錄},
    figurename={圖},
    tablename={表},
    abstractname={摘要},
    indexname={索引},
    appendixname={附錄},
    bibname={參考文獻}
}

% 基本套件
\usepackage{geometry}
\usepackage{setspace}
\usepackage{graphicx}
\usepackage{amsmath}
\usepackage{amsfonts}
\usepackage{amssymb}
\usepackage{cite}
\usepackage{url}
\usepackage{hyperref}

% 頁面設定
\geometry{
    left=3cm,
    right=2cm,
    top=2.5cm,
    bottom=2.5cm
}

% 行距設定
\onehalfspacing

% 標題資訊
\title{船舶群集搜救模擬系統}
\author{作者姓名}
\date{\today}

\begin{document}

% 標題頁
\maketitle

% 摘要
\begin{abstract}
海上搜救任務面臨著範圍廣闊、海象險惡、時間緊迫等嚴峻挑戰。為應對此一難題,本專案建構了一套「船舶群集搜救模擬系統」。系統核心旨在模擬並最佳化真實的搜救作業流程:首先,依據目標可能區域進行高效的網格化分割;接著,指揮調度多艘船隻構成的搜救群集,對各區塊展開平行搜索,以最大化覆蓋率並縮短黃金救援時間。我們在 Unity 與 Crest 物理引擎打造的擬真海洋環境中,採用 SAC 深度強化學習演算法,賦予每艘搜救船隻在複雜風浪中自主執行精密搜索路徑、並規避動態障礙物的能力。本專案旨在評估此 AI 驅動的搜救群集,在不同搜救情境下,對於提升目標發現成功率、縮短搜救時間的實際成效,為未來智慧化海上應急響應系統提供關鍵的模擬驗證。
\end{abstract}

\noindent\textbf{關鍵詞:} 船舶群集、搜救系統、深度強化學習、多智慧體系統、海洋模擬

\newpage

% 目錄
\tableofcontents
\newpage

% 正文開始
\section{緒論}

\subsection{研究背景與動機}
海洋覆蓋地球表面超過七成,是全球貿易、資源開發與休閒活動的核心場域。然而,隨著海上交通與作業日益頻繁,船舶事故、惡劣氣候與人員落水的風險持續上升。一旦發生事故,搜救(Search and Rescue, SAR)行動成為攸關人命的即時挑戰,其中「黃金救援時間」決定了受困者的生存率。

傳統搜救模式仰賴人員經驗與指揮調度,在大範圍海域與高風險環境下常受限於人力與效率。隨著人工智慧(AI)、自主系統(Autonomous Systems)與高擬真模擬技術的成熟,發展基於智慧化與自主化的搜救系統已成為突破現有困境的契機。自主船舶群集可藉由耐航性、即時協作與精確導航,提升搜救任務的速度與覆蓋範圍,進而增加受困人員的獲救機率。本研究即在此背景下展開。

\subsection{問題陳述}
雖然自主船舶與強化學習的應用展現巨大潛力,但在搜救場景中仍面臨多重挑戰:

\begin{enumerate}
\item \textbf{搜索效率不足}:傳統平行線搜索在大範圍海域下難以兼顧效率與覆蓋率。
\item \textbf{環境動態與不確定性}:風浪與洋流使船隻航行與感測受干擾,增加搜尋難度。
\item \textbf{群集協同複雜性}:多艘船隻需避免重複搜索、確保安全並即時共享資訊。
\item \textbf{時間壓力與決策挑戰}:有限時間內需快速分配資源並做出最佳判斷,錯誤可能導致錯失救援時機。
\end{enumerate}

因此,有必要建構一個智慧化模擬系統,以驗證不同策略並最佳化搜救行動。

\subsection{研究目的}
本研究的主要目標在於建立一個「智慧化船舶群集搜救模擬系統」,具體目的如下:

\begin{itemize}
\item \textbf{建立高擬真虛擬環境}:以 Unity 與 Crest 套件模擬真實海象與物理效果,作為測試平台。
\item \textbf{開發自主搜救智慧體}:利用深度強化學習演算法(SAC 為核心,並比較 PPO 等方法),訓練船舶於動態環境中進行自主導航與避障。
\item \textbf{設計群集協同策略}:透過分區搜尋與任務分配,提升多船協同效率,避免重複與衝突。
\item \textbf{效能評估與驗證}:以模擬實驗檢驗系統在搜尋成功率、平均搜救時間與協同效率上的成效。
\end{itemize}

\subsection{專題架構}
本專題共分為七章,內容安排如下:

\begin{itemize}
\item \textbf{第一章 緒論}:說明研究背景、問題陳述、研究目的與章節安排。
\item \textbf{第二章 文獻回顧與討論}:探討海上搜救、強化學習、多代理人系統與模擬相關研究。
\item \textbf{第三章 相關使用技術介紹}:介紹研究所採用之深度強化學習演算法(PPO/SAC)、Unity3D、ML-Agents 與 Crest 模擬套件。
\item \textbf{第四章 系統設計與技術實作}:說明系統硬體與軟體環境、系統架構設計、強化學習架構與套件整合方法。
\item \textbf{第五章 實驗設計與結果分析}:呈現實驗設計、數據收集與結果分析,驗證系統效能。
\item \textbf{第六章 結論與未來發展}:總結研究成果,並提出研究限制與未來可能發展。
\item \textbf{第七章 參考資料}:列出研究所參考之文獻與資料來源。
\end{itemize}

\section{文獻回顧}
% 請在此處添加文獻回顧內容

\section{系統設計與實作}
% 請在此處添加系統設計內容

\section{實驗結果與分析}
% 請在此處添加實驗結果內容

\section{結論與未來工作}
% 請在此處添加結論內容

% 參考文獻
\begin{thebibliography}{9}

\bibitem{example1}
作者姓名,
``論文標題,''
\textit{期刊名稱},
vol.~1, no.~1, pp.~1--10,
月份 年份.

\bibitem{example2}
作者姓名,
``書籍標題,''
出版社名稱,
年份.

% 在此處添加更多參考文獻

\end{thebibliography}

\end{document}