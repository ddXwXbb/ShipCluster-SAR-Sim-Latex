\documentclass[12pt,a4paper]{article}

% 中文支援套件
\usepackage[UTF8,scheme=chinese]{ctex}
\usepackage{xeCJK}

% 繁體中文設定
\ctexset{
    contentsname={目錄},
    listfigurename={圖目錄},
    listtablename={表目錄},
    figurename={圖},
    tablename={表},
    abstractname={摘要},
    indexname={索引},
    appendixname={附錄},
    bibname={參考文獻}
}

% 基本套件
\usepackage{geometry}
\usepackage{setspace}
\usepackage{graphicx}
\usepackage{amsmath}
\usepackage{amsfonts}
\usepackage{amssymb}
\usepackage{cite}
\usepackage{url}
\usepackage{hyperref}

% 頁面設定
\geometry{
    left=3cm,
    right=2cm,
    top=2.5cm,
    bottom=2.5cm
}

% 行距設定
\onehalfspacing

% 標題資訊
\title{船舶群集搜救模擬系統}
\author{作者姓名}
\date{\today}

\begin{document}

% 標題頁
\maketitle

% 摘要
\begin{abstract}
海上搜救任務面臨著範圍廣闊、海象險惡、時間緊迫等嚴峻挑戰。為應對此一難題,本專案建構了一套「船舶群集搜救模擬系統」。系統核心旨在模擬並最佳化真實的搜救作業流程:首先,依據目標可能區域進行高效的網格化分割;接著,指揮調度多艘船隻構成的搜救群集,對各區塊展開平行搜索,以最大化覆蓋率並縮短黃金救援時間。我們在 Unity 與 Crest 物理引擎打造的擬真海洋環境中,採用 SAC 深度強化學習演算法,賦予每艘搜救船隻在複雜風浪中自主執行精密搜索路徑、並規避動態障礙物的能力。本專案旨在評估此 AI 驅動的搜救群集,在不同搜救情境下,對於提升目標發現成功率、縮短搜救時間的實際成效,為未來智慧化海上應急響應系統提供關鍵的模擬驗證。
\end{abstract}

\centerline{\textbf{關鍵詞:} 船舶群集、搜救系統、深度強化學習、多智慧體系統、海洋模擬}

\newpage

% 目錄
\tableofcontents
\newpage

% 正文開始
\section{緒論}

\subsection{研究背景與動機}
海洋覆蓋地球表面超過七成,是全球貿易、資源開發與休閒活動的核心場域。然而,隨著海上交通與作業日益頻繁,船舶事故、惡劣氣候與人員落水的風險持續上升。一旦發生事故,搜救(Search and Rescue, SAR)行動成為攸關人命的即時挑戰,其中「黃金救援時間」決定了受困者的生存率。

傳統搜救模式仰賴人員經驗與指揮調度,在大範圍海域與高風險環境下常受限於人力與效率。隨著人工智慧(AI)、自主系統(Autonomous Systems)與高擬真模擬技術的成熟,發展基於智慧化與自主化的搜救系統已成為突破現有困境的契機。自主船舶群集可藉由耐航性、即時協作與精確導航,提升搜救任務的速度與覆蓋範圍,進而增加受困人員的獲救機率。本研究即在此背景下展開。

\subsection{問題陳述}
雖然自主船舶與強化學習的應用展現巨大潛力,但在搜救場景中仍面臨多重挑戰:

\begin{enumerate}
\item \textbf{搜索效率不足}:傳統平行線搜索在大範圍海域下難以兼顧效率與覆蓋率。
\item \textbf{環境動態與不確定性}:風浪與洋流使船隻航行與感測受干擾,增加搜尋難度。
\item \textbf{群集協同複雜性}:多艘船隻需避免重複搜索、確保安全並即時共享資訊。
\item \textbf{時間壓力與決策挑戰}:有限時間內需快速分配資源並做出最佳判斷,錯誤可能導致錯失救援時機。
\end{enumerate}

因此,有必要建構一個智慧化模擬系統,以驗證不同策略並最佳化搜救行動。

\subsection{研究目的}
本研究的主要目標在於建立一個「智慧化船舶群集搜救模擬系統」,具體目的如下:

\begin{itemize}
\item \textbf{建立高擬真虛擬環境}:以 Unity 與 Crest 套件模擬真實海象與物理效果,作為測試平台。
\item \textbf{開發自主搜救智慧體}:利用深度強化學習演算法(SAC 為核心,並比較 PPO 等方法),訓練船舶於動態環境中進行自主導航與避障。
\item \textbf{設計群集協同策略}:透過分區搜尋與任務分配,提升多船協同效率,避免重複與衝突。
\item \textbf{效能評估與驗證}:以模擬實驗檢驗系統在搜尋成功率、平均搜救時間與協同效率上的成效。
\end{itemize}

\subsection{專題架構}
本專題共分為七章,內容安排如下:

\begin{itemize}
\item \textbf{第一章 緒論}:說明研究背景、問題陳述、研究目的與章節安排。
\item \textbf{第二章 文獻回顧與討論}:探討海上搜救、強化學習、多代理人系統與模擬相關研究。
\item \textbf{第三章 相關使用技術介紹}:介紹研究所採用之深度強化學習演算法(PPO/SAC)、Unity3D、ML-Agents 與 Crest 模擬套件。
\item \textbf{第四章 系統設計與技術實作}:說明系統硬體與軟體環境、系統架構設計、強化學習架構與套件整合方法。
\item \textbf{第五章 實驗設計與結果分析}:呈現實驗設計、數據收集與結果分析,驗證系統效能。
\item \textbf{第六章 結論與未來發展}:總結研究成果,並提出研究限制與未來可能發展。
\item \textbf{第七章 參考資料}:列出研究所參考之文獻與資料來源。
\end{itemize}

\newpage

\section{文獻回顧與討論}

\subsection{海上搜救相關研究}

\subsubsection{傳統海上搜救方法}
\paragraph{Expanding square search(擴展方形搜尋)}  
適用於目標位置大致已知、中等範圍的搜尋。搜尋點從基準點開始搜尋,每次往前一段距離後轉向 90 度,每次轉向航段都會逐漸加長。需要準確的導航系統。

\paragraph{Sector search(扇形搜尋)}  
適用於目標位置明確、小範圍的搜尋。搜尋方式由基準點為圓心,進行放射狀的搜尋。最多只能由一艘船與一架飛機分別進行搜尋。

\paragraph{Parallel sweep search(平行航線搜尋)}  
適用於目標位置未知、大範圍的搜尋。搜尋區域被劃分為多個子區域,由不同的船舶分工搜尋,每個子區域的搜尋航線互相平行,並與海流方向一致。由於派出的單位較多,因此需要消耗大量的資源。

\subsubsection{傳統方法的限制與挑戰}
\begin{itemize}
  \item \textbf{安全風險}:傳統的搜救方式需要出動大量的人員與船舶,如在極端天候可能會造成額外的風險。
  \item \textbf{人力與資源消耗}:以平行航線為例,雖然能夠覆蓋較大的範圍,但同時也需要使用大量的船舶及人員,而搜救成本也隨之增加。
  \item \textbf{搜尋效率受限}:擴展正方形、扇形搜尋雖然成本相較為低廉,然而其搜尋範圍相對有限,並且在海流及海風的影響下,航線容易偏移,降低搜尋效率。
  \item \textbf{時效性不足}:當目標位置未知且搜尋範圍過大,傳統搜尋方法往往很難在黃金救援期間完成全面搜尋。
\end{itemize}

\subsubsection{技術發展}
隨著科技進步,傳統搜救方式的限制逐漸被新技術所補足:
\begin{itemize}
  \item \textbf{無人機}:具備快速部屬及高機動性,能在短時間內對大範圍的海域進行空中偵查。
  \item \textbf{自主船舶}:可執行長時間、低成本的搜尋,降低人員風險與油料消耗。使用群集作業可以補足平行航線搜尋的缺點,在維持大範圍的情況下同時減少搜尋成本。
  \item \textbf{混和搜救系統}:結合有人、無人船舶及無人機,形成「立體化」的搜尋網路。有人單位可以集中精力於高價值決策或特殊任務,無人單位負責長時間及大範圍的搜尋。
\end{itemize}

\subsection{自主船舶系統現代應用}
自主船舶(USV)的發展:隨著無人化技術的推進,USV 在海洋產業中逐漸成熟並投入實際應用。主要集中在以下的面向:
\begin{itemize}
  \item \textbf{任務自動化與導航精準度}:透過高精度 GPS 與自主導航演算法,使船舶能在無人操控下完成大範圍的搜尋與調查。
  \item \textbf{能源與續航力優化}:積極推動電力及太陽能的使用,用以滿足長時間海上作業的需求。
  \item \textbf{安全與可靠性}:增設船體健康監控、故障預警與各種感測器,以即時偵測船體外部環境進行自主避障。
  \item \textbf{應用多元化}:涵蓋貨運、離岸風電運維、海洋調查與急難救助等。
\end{itemize}

\paragraph{相關應用案例}
\begin{itemize}
  \item Yara Birkeland:全球首艘全電動無人貨櫃船,展現大型自主船舶的商轉潛力。
  \item 日本 MEGURI 2040 計畫:推動多艘無人渡輪與貨船運行,測試長距離自主航行能力。
  \item 英國 Rolls-Royce、韓國現代重工:積極投入智慧船舶與自主航行技術。
\end{itemize}

\subsection{深度強化學習於自主導航領域之研究}

\subsubsection{傳統演算法與限制}
在自主導航(Autonomous Navigation)領域中,傳統的路徑規劃方法如 A*、Dijkstra 與快速隨機樹(Rapidly-exploring Random Tree, RRT),一直是最為基礎且廣泛使用的演算法。

\begin{itemize}
  \item \textbf{Dijkstra}:以圖論為基礎,透過計算從起點到所有節點的最短距離來確保最短路徑的正確性。此方法在靜態地圖或事先已知環境中具備嚴謹性,但計算複雜度高,且在動態情境下缺乏快速更新能力,難以應對環境中即時出現的障礙物或突發狀況。
  \item \textbf{A*}:結合啟發式函數與 Dijkstra 的最短路徑搜尋概念,能在靜態環境下有效找到近似最優路徑。其優點在於搜尋效率高,且能保證找到全域最優解。然而在動態環境中,當障礙物或目標位置發生變化時,A* 必須重新進行全域規劃,導致即時性不足。
  \item \textbf{RRT}:屬於取樣式(sampling-based)的規劃方法,能在高維空間中迅速探索可行路徑,特別適合處理複雜的路線規劃。然而其生成的路徑往往並非最優,且需要額外優化程序。同時,RRT 在面對動態障礙物時也需頻繁重新生成樹狀結構,導致計算負擔增加。
\end{itemize}

\noindent 綜觀而言,傳統演算法的主要限制包括:
\begin{itemize}
  \item 缺乏即時適應性:當環境發生變化(如動態障礙物、移動目標)時,需重新規劃路徑,反應速度不足。
  \item 計算負擔大:在高維、複雜或大規模環境下,演算法的計算時間與記憶體需求急遽增加。
  \item 難以處理不確定性:傳統規劃多依賴完整或靜態的環境資訊,在感測器噪音、動態行人或其他不確定因素下,效能顯著下降。
\end{itemize}

因此,雖然傳統方法在靜態場景中表現穩健,但在動態且不確定的自主導航環境中,往往無法滿足即時決策與適應性需求,這正是深度強化學習(Deep Reinforcement Learning, DRL)開始受到重視的原因。


\subsubsection{深度強化學習的突破}
為克服傳統演算法在動態環境下的不足,深度強化學習(Deep Reinforcement Learning, DRL)成為自主導航研究中的重要突破。與 A*、Dijkstra、RRT 等需要完整環境模型並重新規劃的方式不同,DRL 採用 trial-and-error(嘗試錯誤)學習機制,透過與環境的交互不斷更新策略(policy),進而實現動態決策與即時反應。

\paragraph{(1) 自主學習與即時決策}
DRL 將強化學習(Reinforcement Learning, RL)的價值函數或策略函數,結合深度神經網路的表徵能力,使智能體能在高維感知輸入(如 LiDAR、相機影像)下直接學習導航策略。這種方式無需依賴精確的地圖建模或完整環境資訊,而是透過即時觀測與回饋,逐步形成能適應不同情境的決策能力。

\paragraph{(2) 處理動態與不確定性}
相較於傳統演算法須頻繁重新規劃路徑,DRL 在訓練過程中即已接觸到各種動態障礙物與隨機擾動,因而具備在未知或變化環境中快速調整的能力。例如,當遇到突然出現的障礙物時,智能體能即時採取繞行或減速等動作,而不必從頭運算整條路徑。

\paragraph{(3) 泛化與適應性}
透過大量訓練資料與模擬環境,DRL 模型能學會在不同場景間泛化。例如,同一策略可在室內走廊、室外道路或不同地圖配置中應用,具備比傳統演算法更強的靈活性。這使得 DRL 特別適合應用於具有高度不確定性的場景,如自駕車或自主移動機器人。

\noindent 綜合而言,DRL 透過嘗試錯誤式的學習框架,彌補了傳統演算法在動態性、即時性與不確定性上的不足,為自主導航帶來全新的解決方案。這些優勢使其在自駕車、無人機以及服務型機器人等領域展現出廣泛的應用潛力。

\subsubsection{核心演算法應用}
\begin{itemize}
  \item \textbf{DQN(Deep Q-Network)}:為傳統 Q-Learning 的延伸,利用深度神經網路來近似 Q-Value,解決傳統 Q-Table 無法應付大型或連續狀態空間的問題。
  \item \textbf{PPO(Proximal Policy Optimization)}:一種在策略空間進行優化的演算法,核心概念是在保證新的策略與舊的策略差異不會太大的前提下,尋找一個性能更好的策略。
  \item \textbf{SAC(Soft Actor-Critic)}:基於 Actor-Critic 的演算法,融合最大熵強化學習,在學習高回報策略的同時,鼓勵策略保持隨機性以提升探索能力,適合高維度與連續動作空間。
\end{itemize}

\paragraph{案例分析}
\begin{itemize}
  \item 無人車模擬環境中的 DQN 應用:一項研究探討了 DQN 在無人車模擬環境中的應用,特別是在安全、正常與激進駕駛模式下的表現。結果顯示,DQN 能夠有效學習並適應不同駕駛模式。
  \item 無人機應急導航:斯坦福大學的研究團隊開發了一個基於 DQN 和 SAC 的無人機導航系統,用於應急情境中的快速部署與導航。
\end{itemize}

\subsection{模擬平台與方法比較}

\begin{itemize}
  \item \textbf{模擬平台比較}:
  為了模擬現實自主船舶在大範圍海域的搜救能力,本專案需一個能夠模擬大範圍海域、高擬真波浪、同時支援運行自主航行演算法的測試平台。主要平台比較如下:
  \begin{itemize}
    \item \textbf{Unity}:跨平台、視覺效果強、支援 Crest 波浪插件。限制在物理模擬精度相較專業模擬器稍差。適用於大規模海洋場景、波浪及執行群集航行策略演算法。
    \item \textbf{Gazebo}:與 ROS 深度整合、演算法測試方便。限制在大規模海洋場景支援有限,難以呈現複雜海況。
    \item \textbf{MORSE (Modular OpenRobots Simulation Engine)}:研究導向,支援機器人學術模擬與感測器模擬。限制為社群小、更新頻率低、擴充性較差,僅適合學術性演算法測試。
    \item \textbf{Webots}:教育與研究常用,內建機器人模型與控制介面。限制在大範圍海洋模擬上表現不足,適合小型場景。
  \end{itemize}
  
  \item \textbf{Unity 與 Crest 的優勢}:
  \begin{itemize}
    \item \textbf{真實感佳}:Crest 基於 FFT 波譜法生成海浪,模擬效果逼真。
    \item \textbf{即時渲染}:支援 GPU 加速,可進行大規模的波浪實時模擬。
    \item \textbf{無縫整合}:Crest 作為 Unity 插件,不需要額外轉換資料或外部軟體支援。
  \end{itemize}
  綜合比較:Unity + Crest 能夠同時滿足海洋擬真、模擬規模及自主船舶的控制測試需求,因此我們選擇其作為首選模擬環境。
\end{itemize}

\subsection{代理人系統}

\begin{itemize}
  \item \textbf{單代理人系統(Single-Agent Systems, SAS)}:
  單代理人系統是指整個任務由一個代理人獨立完成。適用於任務範圍小、環境單純的情境,成本較低廉,但在大範圍及複雜環境下效率受限。
  
  \item \textbf{多代理人系統(Multi-Agent Systems, MAS)}:
  MAS 由多個自主代理人組成,每個代理人能獨立運作,並透過協作、競爭等方式完成各自或共同的任務目標。常應用於大規模、分散且動態的環境。其訓練難度較高,主要挑戰包括:
  \begin{itemize}
    \item \textbf{動態環境}:每個代理人在學習中會改變策略,導致環境不斷變化,難以捉摸。
    \item \textbf{部分觀測與不確定性}:每個代理人只能感知局部環境資訊,需要推斷其他代理人的狀態及意圖。
    \item \textbf{協作與衝突}:代理人需要學習如何協作及分工,避免互相干擾。
  \end{itemize}
\end{itemize}

\subsection{文獻總結與研究方向}

\begin{itemize}
  \item \textbf{研究缺口}:缺乏針對「群集自主船舶」在惡劣環境下進行搜救的高擬真模擬平台。
  \item \textbf{本研究貢獻}:
  \begin{itemize}
    \item 建立結合 Unity 與 Crest 的海洋高擬真測試平台。
    \item 開發基於 SAC 的自主船舶智慧體,並對比 PPO 等方法。
    \item 提出可行的群集協同策略,並透過模擬驗證效能。
  \end{itemize}
\end{itemize}


\subsection{文獻總結與研究方向}
研究缺口、本研究貢獻。


\section{相關使用技術介紹}
% 請在此處添加系統設計內容

\section{系統設計與技術實作}
% 請在此處添加實驗結果內容

\section{實驗設計與結果分析}
% 請在此處添加結論內容

\section{結論與未來發展}

% 參考文獻
\begin{thebibliography}{9}

\bibitem{example1}
作者姓名,
``論文標題,''
\textit{期刊名稱},
vol.~1, no.~1, pp.~1--10,
月份 年份.

\bibitem{example2}
作者姓名,
``書籍標題,''
出版社名稱,
年份.

% 在此處添加更多參考文獻

\end{thebibliography}

\end{document}